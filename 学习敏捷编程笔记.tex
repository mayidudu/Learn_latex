\documentclass{ctexart}
\CTEXsetup[format={\Large\bfseries}]{section}
\newcommand{\jn}{JN}
\begin{document}

\title{学习笔记--敏捷编程}
\author{mayidudu 来自 \jn}
\maketitle
%\tableofcontents

\section{敏捷原则}
最优先要做的是尽早、持续地交付有价值的软件

欣然面对需求变化,即使是在开发后期

频繁交付可工作的软件,交付周期越短越好

可工作的软件是衡量进度的首要标准

倡导可持续开发

坚持不懈追求技术卓越和设计卓越

简单是尽最大可能减少不必要的工作

定期反思如何提升效率


\section{Scrum和自组织团队}

适用条件

\section{极限编程}
\subsection{编程实践}
测试先行编程和结对编程
\subsection{集成实践}
10-minute build,开发团队需要一个自动构建全部代码的机制,而且完成自动构建的实践不超过10分钟。
构建过程包括运行所有的单元测试并生成一个报告。

持续集成,代码仓库,允许多人在同一份代码进行修改。

\subsection{计划实践}
周循环(weekly cycle)
每个周期,以一个会议开始,会上团队成员回顾目前进度(1),确定本周迭代的目标(2),分解成具体的任务(3),对任务的进行工作量的估计(4)并分配到具体的开发人员(5)。

季度循环

丢车保帅

\subsection{团队实践}
面对面

大信息量的工作空间

\section{简化和增量式设计}

今天看似美好的框架,明天就可能成为负担

更早地部署软件

做决定的时候更有把握

在原有设计前提发生变化时保持开发速度

持续集成,排查设计问题

避免一体式设计

编写小型的、可靠的、独立的单元组成的松耦合代码的习惯

简化单元交互,系统实现增量式成长;
优秀的设计源自简单的交互


\section{精益思维}
精益的价值观
\begin{enumerate}
\item  {消除浪费}
\item {增强学习}
\item {尽可能延迟决定}
\item {尽快交付} 
\item {帮助团队成功} 
\item {保证产品完善} 
\item {着眼全局}
\end{enumerate}

精益+极限+Scrum,三者的交汇点是:最后责任时刻,增强学习,精力充沛地工作,专注。

\end{document}
