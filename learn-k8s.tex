\documentclass{ctexart}
\CTEXsetup[format={\Large\bfseries}]{section}
\newcommand{\jn}{JN}
\usepackage{graphicx}
\usepackage{float}
\usepackage{listings}
\begin{document}

\title{学习笔记--k8s}
\author{mayidudu 来自 \jn}
\maketitle
%https://blog.csdn.net/wcx1293296315/article/details/79775671, 插图
\tableofcontents


\section{架构与组成部分}

\section{基本概念}
\begin{enumerate}
	\item [-] Master
	
	\item [-] Node
	
	\item [-] Pod
	
	\item [-] Label
	
	\item [-] Replication Controller (RC)
	
	\item [-] Deployment
	
	\item [-] Horizontal Pod Autoscaler(HPA)
	
	\item [-] Service
	
	\item [-] Volume
	
	\item [-] Persistent Volume
	
	\item [-] Namespace
	
	\item [-] Annotation
	
	\item [-] Others
	
	
\end{enumerate}
\section{基本命令}
\lstset{language=C}
\begin{lstlisting}
$kubectl [command] [TYPE] [NAME] [flags]
\end{lstlisting}
\subsection{子命令}
\begin{table}[H]
	\centering
	\caption{命令}
	\begin{tabular}{|l|p{0.9\columnwidth}|}
		\hline
		 create & \\\hline
	delete & \\\hline
	edit & \\\hline
		describe& \\\hline
	get& \\\hline
	label & \\\hline
	logs & \\\hline
	namespace & \\\hline
	patch & \\\hline
	port-forward & \\\hline
	proxy & \\\hline
	replace & \\\hline
	rolling-update & \\\hline
	rollout & \\\hline
	run & \\\hline
	scale & \\\hline
	set & \\\hline
	taint & \\\hline
	uncordon & \\\hline
	version & \\\hline
	apply& \\\hline
	exec & 执行容器中的命令\\\hline
	explain & \\\hline
annotate & 添加或更新资源对象的annotation信息 \\\hline
api-version& \\\hline
attach & kubectl attach POD -c CONTAINER [flag]  \\\hline
autoscale & 对deployment,replicaset,replicationController进行水平扩容  \\\hline
cluster-info & \\\hline
completion & kubectl completion SHELL [flag] 输出shell命令的执行结果码 \\\hline
config & 修改config文件  \\\hline
convert & 将配置文件在不同API版本中进行转换\\\hline
cordon & 将Node标记为不可调度,即隔离 \\\hline
drain & 先将Node设置为unschedulable,然后删除该node运行的所有Pod,但是不会删除不由apiserver管理的pod。 \\\hline
%
	\end{tabular}
\end{table}


\subsection{参数}
\begin{table}[H]
	\centering
	\caption{参数列表}
	\begin{tabular}{|l|c|}
		\hline
		参数名 & 说明 \\\hline
		--alsologtostderr[=false] & \\\hline
		--logtostderr[=true] & \\\hline
		--as="" & 设置本次操作的用户名 \\\hline
		--certificate-authority="" & \\\hline
		--client-key="" & \\\hline
		--cluster="" & \\\hline
		--context="" & \\\hline
		--insecure-skip-tls-verify[=false] & \\\hline
		--kubeconfig="" & \\\hline
		--log-backtrace-at=:0 & \\\hline
		--log-dir="" & \\\hline
		--log-flush-frequency=5s & \\\hline
		--match-server-version[=false] & \\\hline
		--namespace="" & \\\hline
		--password="" & \\\hline
		-s,--server="" & apiserver url \\\hline
		--stderrthreshold=2 & \\\hline
		--token="" & \\\hline
		--user="" & 指定kubeconfig 的用户名 \\\hline
		--username="" & apiserver的用户名 \\\hline
		--v=0 & glog 日志级别 \\\hline
		--vmodule= & glog 基于模块的详细日志级别 \\\hline
		
	\end{tabular}
\end{table}

\subsection{输出格式}
\begin{table}[H]
	\centering
	\caption{命令输出格式}
	\begin{tabular}{|l|c|}
		\hline
		输出格式 & 说明\\\hline
		-o=custom-columns=<spec> & \\\hline
		-o=custom-columns-file=<filename> & \\\hline
		-o=json & \\\hline
		-o=jsonpath=<template> & \\\hline
		-o=jsonpath-file=<filename> & \\\hline
		-o=name & \\\hline
		-o=wide & \\\hline
		-o=yaml & \\\hline
	\end{tabular}
\end{table}



\subsection{资源类型}
区分大小写,能以单数、复数或简写形式表示


\begin{enumerate}
	\item [-] componentstatuses(cs)
	\item [-] daemonsets(ds)
	\item [-] deployments
	\item [-] events(ev)
	\item [-] endpoints(ep)
	\item [-] horizontalpodautoscalers(hpa)
	\item [-] ingresses(ing)
	\item [-] jobs
	\item [-] limitranges(limits)
	\item [-] nodes(no)
	\item [-]namespaces(ns)
	\item [-] pods(po)
	\item [-] persistentvolumes(pv)
	\item [-] persistentvolumeclaims(pvc)
	\item [-] resourcequotas
	\item [-] replicationcontrollers(rc)
	\item [-] secrets
	\item [-] serviceaccounts
	\item [-] services(svc)
	
	
\end{enumerate}


\section{应用方向}



\end{document}